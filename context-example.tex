\setupoutput[pdftex]
\setupbodyfont[plr,11pt]
\mainlanguage[de]
\language[de]
\enableregime[utf]

\setuphead[title][style={\ss\bfd},
% hier wird das Aussehen des Titels definiert
    before={\begingroup},
    after={Hans Wikipedianer\bigskip\endgroup}]

\starttext

\title{\ConTeXt}

\section{Text}
Genau so wie \LaTeX{} macht es auch \ConTeXt{}
einfach, den Text durch eine
Abschnittsnummerierung sowie durch Verweise
auf Tabellen, Zeichnungen und andere Elemente
zu gliedern. So kann man ganz einfach auf
Gleichung \in[eqn:gleichung100] verweisen.

\section{Mathematik}
Die folgende Gleichung stellt die
Möglichkeiten von \ConTeXt{} auf dem Gebiet
mathematischer Formeln dar. Gleichungen können
automatisch nummeriert werden.
\placeformula[eqn:gleichung100]
\startformula
    E = mc^2
\stopformula
worin
\placeformula[eqn:gleichung200]
\startformula
    m = \frac{m_0}{\sqrt{1-\frac{v^2}{c^2}}}
\stopformula
ist.

\stoptext
